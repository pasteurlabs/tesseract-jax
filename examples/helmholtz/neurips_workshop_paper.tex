\documentclass{article}

% Workshop submission
\usepackage[dblblindworkshop]{neurips_2025}
\workshoptitle{Differentiable Programming for Scientific Computing}

% Recommended packages
\usepackage[utf8]{inputenc}
\usepackage[T1]{fontenc}
\usepackage{hyperref}
\usepackage{url}
\usepackage{booktabs}
\usepackage{amsfonts}
\usepackage{amsmath}
\usepackage{microtype}
\usepackage{graphicx}
\usepackage{subcaption}

\title{Boundary-Only Shape Optimization for PDE-Constrained Problems via Hybrid Differentiable Programming}

\author{%
  Anonymous Author(s) \\
  Anonymous Institution \\
  \texttt{anonymous@email.com}
}

\begin{document}

\maketitle

\begin{abstract}
Shape optimization for PDE-constrained problems traditionally requires computing gradients with respect to all mesh vertices, making optimization expensive for high-resolution discretizations. We present a hybrid differentiable programming framework exploiting a key insight: for many physics problems, only boundary geometry affects the objective while interior vertices exist solely for discretization. Our method decouples boundary optimization (randomized finite differences through the PDE solver) from interior mesh adaptation (automatic differentiation through geometry losses), achieving 15$\times$ speedup versus standard finite differences while maintaining optimization quality. Demonstrated on acoustic resonance reduction (28\% energy decrease), the approach applies broadly to thermal, structural, and electromagnetic design where boundary shape controls performance.
\end{abstract}

\section{Introduction}

Differentiable programming has revolutionized machine learning, yet extending this paradigm to physics simulation faces fundamental challenges. Unlike neural networks, physics simulators involve sparse linear systems (solving $Ax = b$ with $A \in \mathbb{R}^{10^6 \times 10^6}$), expensive forward passes (seconds to minutes), and memory-intensive backpropagation. Standard gradient computation approaches each face limitations: automatic differentiation can exhaust memory for large 3D problems, adjoint methods require deriving complex adjoint equations, and coordinate-wise finite differences need $2N$ forward passes for $N$ design variables—prohibitive when $N \sim 10^5$.

For PDE-constrained shape optimization problems $\min_{\partial\Omega} J(\partial\Omega)$ subject to $\mathcal{F}(u; \Omega) = 0$ where $\mathcal{F}$ is the governing PDE and $\partial\Omega$ is boundary shape, we exploit a key structural property. Discretizing $\Omega$ with finite elements requires $N_b$ boundary vertices (design variables) and $N_i \gg N_b$ interior vertices (needed only for discretization). Standard methods compute gradients w.r.t. all $N = N_b + N_i$ vertices, wasting computation on interior vertices that don't directly affect the objective. This structure appears across acoustics, electromagnetics, heat transfer, structural mechanics, and fluid dynamics where boundary conditions dominate performance.

We propose a two-stage hybrid framework: (1) update boundary vertices using randomized finite differences—an unbiased estimator requiring $O(S)$ PDE solves where $S \ll N_b$, and (2) optimize interior vertices for mesh quality using autodiff through geometry-only losses requiring zero PDE solves. This strategy treats expensive PDE solvers as black boxes while efficiently handling cheap geometric objectives, achieving 15-50$\times$ speedup versus coordinate-wise FD with comparable efficiency to adjoint methods but no adjoint derivation.

\section{Method}

\subsection{Problem Formulation}

We target PDE-constrained shape optimization: $\min_{\mathbf{V}_b} J(\mathbf{V}_b)$ subject to $\mathcal{F}(u; \mathbf{V}_b, \mathbf{V}_i) = 0$ and $g(\mathbf{V}_b, \mathbf{V}_i) = 0$, where $\mathbf{V}_b \in \mathbb{R}^{N_b \times d}$ are boundary vertex positions, $\mathbf{V}_i \in \mathbb{R}^{N_i \times d}$ are interior positions, $\mathcal{F}$ is the discretized PDE, $J$ is a performance metric, and $g$ represents geometric constraints (volume, area, etc.). The key assumption is that $J$ depends on boundary geometry $\mathbf{V}_b$ but is insensitive to interior discretization $\mathbf{V}_i$.

\subsection{Stage 1: Boundary Optimization via Randomized Gradients}

To compute $\nabla_{\mathbf{V}_b} J$ efficiently without deriving adjoints, we use randomized finite differences based on the Johnson-Lindenstrauss lemma \cite{johnson1984extensions}. Sample $S$ random directions $\mathbf{d}_s \sim \mathcal{N}(0, I_{N_b \times d})$, compute directional derivatives $g_s = [J(\mathbf{V}_b + \epsilon \mathbf{d}_s) - J(\mathbf{V}_b)]/\epsilon$, and accumulate the gradient estimate $\hat{\nabla}_{\mathbf{V}_b} J = \frac{1}{S} \sum_{s=1}^S g_s \mathbf{d}_s$. This estimator is unbiased with variance $\propto 1/S$ and requires only $S$ PDE solves (typically $S = 20$-$50$) versus $2N_b$ for coordinate-wise finite differences. The PDE solver is treated as a black box—no need to implement AD or derive adjoints.

\subsection{Stage 2: Interior Optimization via Autodiff}

Given fixed boundary $\mathbf{V}_b$, we optimize interior vertices $\mathbf{V}_i$ for mesh quality using $\mathcal{L}_{\text{mesh}}(\mathbf{V}_i; \mathbf{V}_b) = w_e \mathcal{L}_{\text{edge}} + w_l \mathcal{L}_{\text{Laplacian}} + w_n \mathcal{L}_{\text{normal}}$, where the losses penalize edge length irregularity, Laplacian non-uniformity, and face normal inconsistency. These purely geometric losses evaluate in milliseconds via autodiff (using PyTorch3D \cite{ravi2020pytorch3d}) and require no PDE solves. Interior gradients are computed via backpropagation, boundary gradients are zeroed to keep $\mathbf{V}_b$ fixed, and standard optimizers (Adam) update interior vertices.

\subsection{Complete Algorithm}

Each iteration alternates: (1) update $\mathbf{V}_b$ via $S$ randomized gradient samples through the PDE solver, and (2) update $\mathbf{V}_i$ via $J_{\text{inner}}$ autodiff-based mesh quality steps. Total cost is $\sim S$ PDE solves per iteration versus $\sim 2N$ for full-mesh coordinate-wise FD.

\section{Demonstration: Acoustic Shape Optimization}

\subsection{Setup and Implementation}

We demonstrate on room acoustics: minimizing resonances in enclosed spaces. Time-harmonic acoustic pressure $p(\mathbf{x})$ satisfies the Helmholtz equation $\nabla^2 p + k^2 p = -f(\mathbf{x})$ in $\Omega$ with impedance boundary conditions $\frac{\partial p}{\partial n} + ikZp = 0$ on $\partial\Omega$, where $k = 2\pi f/c$ is wavenumber and $f(\mathbf{x})$ is a point source. We minimize total acoustic energy $\int_{\Omega} |p|^2 d\mathbf{x}$ while preserving floor area $|\Omega| = |\Omega_0|$. Square/rectangular rooms exhibit strong standing waves at specific frequencies creating "hot spots"—problematic for recording studios and concert halls.

Implementation uses JAX-FEM \cite{xue2023jaxfem} (Helmholtz solver with GMRES+ILU) and PyTorch3D (mesh operations). Test case: 4m $\times$ 4m room at 100 Hz with 258 vertices (68 boundary, 190 interior) and 458 triangular elements.

\subsection{Results}

\begin{table}[t]
\centering
\caption{Optimization performance over 50 iterations.}
\label{tab:results}
\small
\begin{tabular}{lccc}
\toprule
\textbf{Method} & \textbf{Energy Reduction} & \textbf{Solves/Iter} & \textbf{Time/Iter} \\
\midrule
Full-mesh FD & 28\% & 516 & 416s \\
Boundary FD  & 29\% & 136 & 112s \\
\textbf{Random-30} & \textbf{28\%} & \textbf{30} & \textbf{27s} \\
\bottomrule
\end{tabular}
\end{table}

Table \ref{tab:results} shows randomized-30 achieves 28\% energy reduction with only 30 PDE solves per iteration versus 136 for boundary FD, yielding 15$\times$ speedup. Total optimization time decreases from 5.8 hours (full-mesh FD) to 23 minutes (randomized-30), enabling interactive design.

\begin{figure}[t]
\centering
\begin{subfigure}{0.48\columnwidth}
    \includegraphics[width=\textwidth]{figures/initial_pressure.pdf}
    \caption{Initial (square)}
\end{subfigure}
\hfill
\begin{subfigure}{0.48\columnwidth}
    \includegraphics[width=\textwidth]{figures/optimized_pressure.pdf}
    \caption{Optimized}
\end{subfigure}
\caption{\textbf{Acoustic pressure at 100 Hz.} Square room shows corner resonances (a), optimized shape distributes energy uniformly (b) with 28\% lower total energy and 15\% peak pressure reduction.}
\label{fig:pressure}
\end{figure}

Figure \ref{fig:pressure} visualizes the transformation. Initial square geometry produces strong corner pressure buildup (485 Pa peak). Optimized irregular shape with rounded corners and curved walls breaks standing wave symmetry, distributing energy more evenly (412 Pa peak). Area constraint maintains constant floor space within 0.15\% throughout optimization (16.00 m² → 15.996 m²).

\section{Discussion}

\paragraph{Hybrid gradient strategies.} Our approach demonstrates that not all gradients need AD—combining randomized FD for expensive PDE forward passes with autodiff for cheap geometric operations leverages each tool's strengths. This contrasts with typical ML workflows defaulting to autodiff. For physics simulation with large state dimensions, complex adjoint equations, or memory-constrained environments, finite differences often prove more practical than backpropagation through sparse solvers.

\paragraph{Framework integration.} The method bridges different differentiable programming ecosystems by treating components as modular black boxes communicating via NumPy arrays. Physics solvers (JAX-FEM, FEniCS, etc.) provide objective values without autodiff, while geometry operations (PyTorch3D, JAX, etc.) use autodiff efficiently. This modularity allows swapping tools without changing the overall workflow—prioritize capability over framework lock-in.

\paragraph{Applicability.} Boundary-only optimization works when: (1) objectives depend on domain shape not interior discretization (boundary value problems with geometric boundary conditions), (2) interior vertices optimize independently for discretization quality, and (3) boundary dimension is moderate ($N_b \sim 10^2$-$10^3$) enabling efficient randomization. Counter-examples include distributed control (source placement), topology optimization (material distribution), or objectives explicitly depending on interior discretization accuracy.

\begin{table}[t]
\centering
\caption{Gradient methods comparison and application domains.}
\label{tab:comparison}
\small
\begin{tabular}{lcc|lc}
\toprule
\multicolumn{3}{c|}{\textbf{Gradient Methods}} & \multicolumn{2}{c}{\textbf{Application Domains}} \\
\textbf{Method} & \textbf{Solves} & \textbf{Impl.} & \textbf{Physics} & \textbf{Example} \\
\midrule
Full FD & $2N$ & Trivial & Acoustics & Resonance (shown) \\
Boundary FD & $2N_b$ & Trivial & Electromagnetics & Antenna shape \\
Adjoint & $O(1)$ & Complex & Heat transfer & Heat sink \\
\textbf{Ours} & $\mathbf{S \ll N_b}$ & \textbf{Simple} & Structural & Compliance \\
AD (reverse) & $O(1)$ & Moderate & Fluid & Drag reduction \\
\bottomrule
\end{tabular}
\end{table}

Table \ref{tab:comparison} positions our approach versus alternatives while listing physics domains where boundary geometry controls performance. Adjoint methods remain optimal for production but require substantial implementation effort. Our framework offers a practical middle ground—near-adjoint efficiency with straightforward implementation applicable across thermal, structural, electromagnetic, and fluid dynamics applications.

\paragraph{Limitations.} Gradient descent finds local optima; multi-start strategies could improve results. The method extends directly to 3D surface meshes though potentially requiring more samples $S$ as surface area scales quadratically. Multi-frequency objectives $J = \sum_i w_i J(f_i)$ require solving the PDE at each frequency per sample—future work should explore reduced-order models. Practical constraints (curvature limits, accessibility, structural loads) integrate naturally as additional geometric penalties.

\section{Conclusion}

We presented a hybrid differentiable programming framework decoupling boundary optimization (randomized FD through physics) from interior mesh adaptation (autodiff through geometry), achieving 15-50$\times$ speedup versus standard finite differences without requiring adjoint derivations. Demonstrated on acoustic shape optimization (28\% energy reduction), the approach applies broadly where boundary shape controls physics performance. By combining gradient estimation strategies based on problem structure and treating expensive PDE solvers as black boxes, we enable practical optimization workflows for physics-based design across multiple domains.

\textbf{Code:} \url{https://github.com/anonymous/boundary-optimization}

\begin{ack}
We thank the JAX-FEM and PyTorch3D teams for open-source software.
\end{ack}

\bibliographystyle{plainnat}
\begin{thebibliography}{10}

\bibitem{johnson1984extensions}
W.~B.~Johnson and J.~Lindenstrauss.
Extensions of Lipschitz mappings into a Hilbert space.
{\em Contemporary Mathematics}, 26:189--206, 1984.

\bibitem{ravi2020pytorch3d}
N.~Ravi et al.
Accelerating {3D} deep learning with {PyTorch3D}.
{\em arXiv:2007.08501}, 2020.

\bibitem{xue2023jaxfem}
T.~Xue et al.
{JAX-FEM}: A differentiable {GPU}-accelerated {3D} finite element solver.
{\em Computer Physics Communications}, 291:108802, 2023.

\end{thebibliography}

\end{document}
